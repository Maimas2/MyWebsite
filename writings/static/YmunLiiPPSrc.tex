\documentclass[12pt]{article}

\usepackage[left=1in, right=1in, top=1in]{geometry}
\usepackage[]{ipsum}
\usepackage[singlespacing]{setspace}
\usepackage[]{hyperref}
\usepackage{textcomp}
\usepackage[]{comment}

\setlength{\parindent}{0pt}
\setlength{\parskip}{8pt}


\pagestyle{empty}

\begin{document}

\begin{center}
	\textbf{\large
		Cardinal Oskar Gruber in Smoke and Silence, YMUN LII\\\vspace*{2pt}
		Alex Seltzer, Waterford High School
	}
\end{center}
\vspace*{-12pt}
The Catholic Church is one of the most influential entities in the world, and we must leverage that power to bring about positive change, not merely ask for others to do so. I, Cardinal Oskar Gruber of Germany, tentatively support Cardinal Augustine Ndlovu for pope, in hopes that he will actively use our church's influence to lead the world towards a better future. To best outline my position on the papal candidates, I will first describe my opinions on the Church's role in the global world, then discuss the election.
\vspace{-8pt}
\begin{center}
	\textbf{\large The Vatican and Global Diplomacy}
\end{center}
\vspace{-14pt}
The Catholic Church, being the oldest extant entity, holds a huge amount of influence over the world and its nations. We hold power to help solve many of the world's problems. To ignore this opportunity would be ignorant and a illogical course of action. As such, the Church must take an active role in global affairs. Even without a vote in bodies such as the United Nations, the Catholic Church can serve as a model for sovereign nations to follow.

Two of my most important issues are the climate crisis and lack of financial ethics. These are two minor issues in the grand scheme of the Church that are likewise likely to fall by the wayside in the debates during and after the Conclave. They are still incredibly important topics that the Church must address as some point. I will try to have the cardinals address them during and after the Conclave.

My driving point is that the Church must actively make a positive impact on the world. This is not to say that this is not already true, but there is always more that the Church can do. In order to exercise the most influence as possible, we must first repair the distrust of the past. The first step of this would be to strengthen Pope Francis' abuse policies. One way to do this could be to make review processes more transparent, for example by publishing a standard method of review to be used across all Catholic dioceses. A top-down method of review, in which Rome plays a larger part in the review process, could ensure that all cases are reviewed roughly the same. Additionally, a global review process could help to prevent offenders who merely leave their original area to escape punishment.

In all world issues, the Church should take a humanitarian standpoint and exclusively aim to help as many people as possible. In wars, we should support organizations such as Doctors without Borders to aid in humanitarian efforts. Crucially, the Church should not take a side in any conflict; for example, in the Russo-Ukranian War, we should simply support aid to both sides. Along with this, I believe that this course of action also calls for not assertively condemning acts of aggression and violence. In order to provide said aid, we do not want any country to feel as if we are working against or undermining their sovereignty. This is especially the case for Russia and China. In the refugee crisis, the Church should work to help all refugees, no matter their past or background. Similar to above, if we do not endorse or condemn either side, our aid may be better received by all parties.


\begin{comment}
	The most pressing issue of the modern world is the climate crisis. We must work on all fronts to combat this threat against our world and our Church. Already, changes to the natural world have caused millions of deaths, and that figure is expected to reach 14.5 million deaths and cause over \$10 trillion in damages by 2050.\footnote{\href{https://www.weforum.org/publications/quantifying-the-impact-of-climate-change-on-human-health/}{https://www.weforum.org/publications/quantifying-the-impact-of-climate-change-on-human-health/}} The changes to climate instituted under Pope Francis were an excellent start, but we can go so much farther. Already, the Vatican City is set to become the first entirely carbon-neutral state in the world after investing about €100 million to source it power entirely from solar panels.\footnote{\href{https://cathnews.com/2025/08/05/vatican-to-become-first-carbon-neutral-state-after-striking-deal-with-italy/}{https://cathnews.com/2025/08/05/vatican-to-become-first-carbon-neutral-state-after-striking-deal-with-italy/}} By further fast-tracking and expanding these plans, the Vatican can help demonstrate to the nations of the world that carbon neutrality is both feasible and effective. The Church could then set ambitious goals to achieve the same in all other parts of the church, to show all governments that carbon neutrality is feasible at both small and large scales.
	
	Another issue which is important to me is the lack of financial ethics in the modern world, especially with the obscurity of public financials. One of the largest problems of today is hidden debt, which is money that a government owes but does not disclose to its people or creditors; the International Monetary Fund (IMF) estimates that there is almost \$1 trillion in hidden debts today.\footnote{\href{https://www.imf.org/en/blogs/articles/2024/04/02/hidden-debt-hurts-economies-better-disclosure-laws-can-help-ease-the-pain}{https://www.imf.org/en/blogs/articles/2024/04/02/hidden-debt-hurts-economies-better-disclosure-laws-can-help-ease-the-pain}} Governments especially are able to hold hide debts due to loopholes or ambiguities in their own public finance laws. The IMF recommends reforming these laws on a country-by-country basis. The Catholic Church can begin modeling financial ethics by revealing all hidden debts, if any, and by improving the---to be blunt---abhorrible record of public transparency. The Vatican published their first ever administrative budget in 1979, and subsequent printings have been sporadic at best.\footnote{Lewin, Ernst A. “The Finances of the Vatican.” Journal of Contemporary History 18, no. 2 (1983): 185–204. \href{http://www.jstor.org/stable/260384}{http://www.jstor.org/stable/260384}.} The first step would be to annually publish these budgets in full; the final step would be a full appraisal of all Church property. By being fully public and transparent, the Catholic Church can encourage the building of financial ethics.
\end{comment}

\pagebreak

\begin{center}
	\textbf{\large Electing the Pope}
\end{center}
\vspace{-14pt}

Before any policy is changed or implemented, we the College of Cardinals must choose a successor to Pope Francis. I have no doubts that every papal candidate would work, back against the current, to lead our Church into the future. Likewise, I will support any cardinal who is willing to compromise and actively bring about change. Based on the candidates' pasts, I feel that Cardinal Augustine Ndlovu would be my preferred choice for the papacy.

As for the place of the Church in the modern world, I believe that we must advocate for peace and humanitarian aid while taking an unequivocally humanitarian role in global conflicts. In the several ongoing wars, we should guarantee aid to all who need it and speak against all forms of violence. Importantly, we should not speak against countries or specific people, only individual acts of violence. By publicly aligning the Church as such, we can help to reaffirm our position as a peaceful and moral voice.

Another method of reaffirming such a position is through further action against clerical abuse. Pope Francis began the reform, and we along with the new pope must finish it. A first step could be to stringently enforce all rules promogulated by the Vatican. When Pope Francis pushed all dioceses to create commissions to combat clerical abuse, many simply chose not to. In Mexico, less than half of dioceses ended up implementing the requested reforms.\footnote{\href{https://losangelespress.org/english-edition/2024/apr/02/less-than-half-the-mexican-catholic-dioceses-prevent-sexual-abuse-8240.html}{https://losangelespress.org/english-edition/2024/apr/02/less-than-half-the-mexican-catholic-dioceses-prevent-sexual-abuse-8240.html}} To counteract against bishops and other Church members who resist reform, the Church should take a top-down approach and require all dioceses to take action. We could also implement a zero-tolerance rule, such that any Church member found to violate abuse guidelines is punished without exception.

I believe that the decentralization of power that occurred under Pope Francis was a mistake. We have seen in the past few years that devolving authority to lower positions only led to a disunification of the Church. As has been seen with issues such as clerical abuse, allowing lower Church officials to take change led to division between different dioceses. This could be solved with a strong papacy and curia to set the tone and contemporary values of the Church.

From these points, it follows that our next pope should take charge of the global Catholic Church to bring about a better future for all through aid and reform. Looking at the beliefs and characteristics of the five candidates, I believe that Cardinal Augustine Ndlovu of South Africa best fits this description, with his previous anti-apartheid action and preaching for racial reconciliation. However, I recognize that every candidate may be willing to come to the table and compromise. Therefore, I cannot give a formal commitment to any papal candidate, not can I rule any out of the running.


\end{document}












