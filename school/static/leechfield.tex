\documentclass[6pt]{bill}

\usepackage[]{comment}
\usepackage[]{lipsum}
\usepackage[none]{hyphenat}

\setlength{\parskip}{0pt}

\raggedright

\begin{document}
	\header{1}{The Leechfield Constitution}{}{}{}{}

	\par Recognizing the need for order and structure to effective democratic system, we the Camp Staff of the Leechfield do hereby certify this Constitution as the governing document of the Leechfield.
	
	\title{}
	
	\section{Committee Structure}{
		\subsection{}{
			The Leechfield Government's power to legislate, enforce, and oversee shall be exclusively vested in a Committee composed of the Camp Staff of the Leechfield.
			\paragraph{}{
				The Committee may, by a vote, decide the name and other identifying emblems or information for the Committee.
			}
		}
		\subsection{}{
			The Committee may, by a vote, decide the time, place, and manner of their meetings.
			\paragraph{}{
				No member of the Leechfield may text or otherwise communicate with their huzz during committee session.
			}
		}
		\subsection{}{
			Each member of the Committee shall receive exactly one vote, and all members' votes shall be in equal standing to each other.
			\paragraph{}{
				This vote is contingent on continued good behavior and sleeping in the Leechfield.
			}
		}
		\subsection{}{
			Any member who moves to a tent outside the Leechfield or permanently leaves camp shall immediately lose their place in the Leechfield Committee, and be refunded a fair amount of money for which they paid into the public fund.
		}
	}
	\section{Committee Leadership}{
		\subsection{}{
			The Committee shall hold elections at will to choose a Mayor to lead the Committee and Leechfield.
			\paragraph{}{
				The Mayor shall act as the figurehead of the Committee and as a liaison between the Committee and higher authorities of camp.
			}
			\paragraph{}{
				The Committee may, by a vote. choose the time, place, and manner, of the elections for Mayor.
			}
		}
		\subsection{}{
			The Committee shall also have a Chair.
			\paragraph{}{
				The Chair shall lead Committee sessions and have the reasonable powers to do so, but otherwise have no additional powers.
			}
		}
		\subsection{}{
			Every person with Committee leadership shall serve at the will of the people. Each may be impeached with a simple motion during committee session and removed from office with a two thirds vote.
		}
	}
	\section{Committee Benefits}{
		\subsection{}{
			The primary beneficiaries shall be the members of the Committee.
		}
		\subsection{}{
			While the Committee may decide to grant benefits to non-members, no non-member may be entitled to the benefits provided by the Committee and membership therein.
		}
	}
	\section{Powers Granted the Committee}{
		\subsection{}{
			The Committee may, by a simple majority, create a small delegation to see that any reasonable task or mission be completed, and grant them the reasonable powers to see it completed.
		}
		\subsection{}{
			The Committee may required reasonable but equal tithes from each its members to form a general fund. The nature of these payments must be agreed to by a unanimous vote.
			\paragraph{}{
				Any member may deny to pay an unreasonably high charge, and provided their denial is reasonable, no member or vote may reverse it.
			}
			\paragraph{}{
				Any member may pay additional money into the public fund of their own volition.
			}
		}
		\subsection{}{
			The Committee may spend the public fund, in a manner decided by a unanimous vote, provided that it is for the betterment of the Leechfield as a whole, not just one person or group.
			\paragraph{}{
				The egregious and intentional theft, misuse, or embezzlement of public funds by any member must be considered grounds for immediate impeachment if possible, expulsion from committee, and referral to higher camp authority.
			}
		}
		\subsection{}{
			The Committee may censure any of its members for gross or egregious conduct by a two thirds vote. They may choose a duration up to three days for which the censured member shall, for all intents and purposes, not be considered a member of committee, except for cases in spending the public fund.
		}
		\subsection{}{
			The Committee may expel any of its members for especially gross or egregious conduct by a unanimous vote.
			\paragraph{}{
				The other members of the Committee may decide a decidedly fair method of trying the accused Committee member.
			}
		}
		
		
		
		\subsection{}{
			The Committee shall have all other necessary and proper powers which are required in exercising the powers listed above.
		}
	}
	\section{Powers Denied the Committee}{
		\subsection{}{
			The Committee may not force any of its members to perform unreasonable, outrageous, or dangerous stunts or activities without the member's consent.
		}
		\subsection{}{
			The Committee may not deny any of its members the equal protection of laws, legislation, regulation, or this Constitution in all committee proceedings, actions, and orders.
		}
		\subsection{}{
			The Committee may not mandate or order the acquisition of any private property for the public Committee or private use.
		}
	}
	
	\title{}
	
	\section{Executive Authority}{
		\subsection{}{
			There shall exist no executive authority within the Leechfield Committee or government.
		}
	}
	\section{Veto Power}{
		\subsection{}{
			There shall exist no power to veto or otherwise unilaterally block any motion or movement within the Committee or its meetings.
		}
	}
	\section{Quorum and Votes}{
		\subsection{}{
			The Committee may not make any lasting or otherwise impactful decisions affecting the Committee, this Constitution, or many Committee members without every member being present to vote.
		}
		\subsection{}{
			All lasting or otherwise impactful votes must be conducted during an in-person Committee meeting.
		}
		\subsection{}{
			The Committee may not open any meeting or make any vote without a simple majority of members present.
		}
		\subsection{}{
			Any member who leaves camp for an extended time shall not restrict the Committee's actions as outlined above.
		}
		\subsection{}{
			Any member may voluntarily waive their absence's restrictions on the Committee's actions outlined above.
		}
		\subsection{}{
			Unless stated otherwise, all votes described in this Constitution shall require a simple majority of all members, present or absent, to agree in order to pass.
		}
	}
	\section{Review of Legislation and Disambiguation}{
		\subsection{}{
			Any member may challenge a law, piece of legislation, or other action as being contrary to this Constitution with a Point of Order during a normal Committee session . The Committee may then choose a fair method and manner of resolving the challenge. If decided to be contrary to this Constitution, that law, piece of legislation, or action shall be immediately null and void and carry no weight or power.
		}
		\subsection{}{
			Any pertinent section of this Constitution with unclear or ambiguous meaning may be brought forth in committee session. By a vote, the Committee members may choose the meaning of this section and clarify it in writing for the future.
		}
	}

	\section{Ratification of this Constitution}{
		\subsection{}{
			This Constitution shall only go into effect only when approved by at least three quarters of the members of the Leechfield.
			\paragraph{}{
				Any person who does not ratify the Constitution shall not be a member of the Committee.
			}
		}
		\subsection{}{
			Any person who moves to the Leechfield after this Constitution's ratification may choose to ratify or reject it. Upon ratification, the new member shall immediately become an member of the Committee with equal standing to all other members and be forced to pay the standard tithe. Upon rejection, the Constitution shall remain in effect for all previous members, and the newcomer shall not be a member of the Committee
		}
		\subsection{}{
			This Constitution shall go out of effect at the end of the camp season, on 15 August 2025.
		}
	}
	
	\section{Amending this Constitution}{
		\subsection{}{
			This Constitution may be reasonably amended by a three quarters vote of the current members. However, no amendment may remove or degrade the restriction on the powers of the Committee, nor grant it unreasonable powers, nor affect any one group of members more negatively than others.
		}
	}
	
	\section{Exiting this Constitution}{
		\subsection{}{
			Any member may voluntarily renounce and otherwise relinquish their membership to the Committee, in which case they are refunded a fair amount of money compared to the amount that they paid into the public fund.
		}
		\subsection{}{
			Any member who relinquishes their membership to the Committee may only reenter the Committee provided that:
			\paragraph{}{
				One day has passed since their renunciation
			}
			\paragraph{}{
				A majority of members agree to their return
			}
			\paragraph{}{
				The member pays into the public fund equal to the amount that they were refunded
			}
		}
	}
	
	\section{Superior Powers}{
		\subsection{}{
			This Constitution, all laws, legislation, and regulations passed by the Committee, and all Committee members are unilaterally subservient to the higher authorities of camp. The ratification of this Constitution shall not be construed to represent any rejection of or departure from such higher authority.
		}
	}
	\eject
	
\end{document}
